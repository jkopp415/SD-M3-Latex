\subsection{Propulsion \& Energy Research Laboratory}

The primary end user of our system will be the researchers at the Propulsion \& Energy Research Laboratory (PERL). The results, findings, and lessons learned of Project SABR will be directly used for further development of this capability. The hardware paid for by the department will remain at PERL for future use, as well.

The goals for this project are to design, build, and test a small-scale air-breathing RDE and its facility, achieve detonation during testing with Hydrogen as the fuel, record load cell data from the test, build an operational map for the combustor, and not sustain significant damage after use. These achievements will be incredibly useful in the characterization of the combustor and will provide insight into the design of the next version and its testing. 

\subsection{Research and Development Programs}

The potential stakeholders of Project SABR are the RDE research and development programs across the country. The ability to study the complicated phenomena associated with RDE operation in air-breathing mode at a fraction of the typical cost will inevitably increase the R\&D activity in this type of propulsion and power generation. The demonstration of a small-scale air-breathing RDE using Hydrogen as fuel will also increase greater interest in realizing this technology.

% TODO: Add reference to section 5 in this paragraph
The stakeholders’ interests in Project SABR include the demonstration of a small-scale RDE using Hydrogen and air, design and demonstration of a unique ignition method for RDEs, the formulation of an operational map of the combustor, and recommendations for future work and designs. After conversations with industry professionals, showing that detonation is possible with air in a small-scale combustor would be incredibly useful to the community. The second most useful outcome of the project would be the novel use of an ignition method in an RDE that would allow for an optimized form factor. This is explored more in Section 5: [Igniter Concept Generation]. The last two products mentioned of the project provide the community with insight into the next problems that need to be tackled for the realization of this technology.

\subsection{Potential Customers}

The potential customers of this system are the U.S. government, NASA, government contractors and other private companies in the aerospace and power generation industries. The inclusion of RDEs in governmental materials and commercial products would decrease the amount of hardware, volume, and weight required for a given work output. Thus, the system itself would cost less, have better performance, and simplify the engineering process (given the current problems at hand are solved). 

The main challenges to overcome with making this a viable product is the reliability of the combustor, the form factor of the supporting systems, the storage of hydrogen fuel, and the cooling of the combustion chamber. There are ongoing and rapidly developing efforts for cooling methods and hydrogen storage methods for operational systems. The reliability of the combustor will be observed during our testing campaign. Our project also aims to explore solutions to the form factor problem with the ignition method.