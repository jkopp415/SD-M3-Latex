\subsection{Injector}

The overarching purpose of an injector is to introduce the propellants into the combustor and mix them as homogeneously as possible. There is not much variation between injectors for RDEs and conventional engines from a fundamental standpoint, as all compressible gas dynamics relations still apply when it comes to orifice sizing and performance. However, certain nuances apply that allow for easier initiation of the detonation wave. Similar to conventional deflagration-based engines, the most basic type of injector geometry is impinging. Impinging injectors feature two streams of propellant colliding into a singular spray cloud, allowing for atomization of liquid propellants, as well as mixing for both gaseous and liquids. The mixing is determined by the angles and relative mass flows of each element. Mixing efficiency is directly connected to relative velocities between the propellants, as the overall goal is to introduce shear between the two with the intent of generating turbulence.

Jet-in-Crossflow injection consists of one of the propellants being injected axially, while the other one is injected radially so the jets meet at 90 degrees from one another. This induces a large amount of mixing for a relatively simple geometry and allows for the propellant chosen to go through the center body possibly being used for regenerative cooling. JIC also allows for partial pre-mixing, where the jets are allowed to meet in a small cavity before transitioning to the annulus.

Mixing efficiency has been experimentally proven to correlate with detonation aspects such as wave speed, wave propagation, as well as conventional parameters such as thrust and specific impulse. Bigler et al. \cite{bigler:2019} demonstrates the effects tied to these parameters, with one of the highlights being that improper mixing leads to counter-rotating waves, which are unfavorable due to their association with lower operating frequencies and wave speeds. These effects are also heavily dependent on local equivalence ratios. A combustor environment with poorly mixed propellants has experimentally been shown to have much more of an impact on wave propagation than the overall equivalence ratio \cite{bigler:2019}. Another alternative injection method is a pre-mixed jet. This method involves mixing the two propellants in a pre-injection chamber or plenum. By mixing upstream of injection, the gases are in contact with each other for a longer period (residence time). The mixing chamber also allows the two gases to reach near stagnation properties while occupying the same space, which results in higher gas-gas diffusion rates and allows for a near uniform mixture before the propellants even enter the chamber. The drawback of this is creating a potentially explosive mixture in the plenum with a high risk of explosion due to the partial backflow characteristic of RDEs.

\vspace{1em}
\noindent The critical aspects for a successful and efficient injector design come down to the following:

\begin{enumerate}[label=(\alph*)]
    \item Well-mixed propellants yield the best benefits in terms of thrust and specific impulse, similar to conventional engines.
    \item Smaller orifices with higher diodicity contribute to better injector replenishment rates, and reduced backflow in the environment directly downstream of the detonation wave.
    \item Choked injectors tend to isolate the detonation waves from the feed system and help increase detonation stability
    \item Efficient injectors yield fewer but stronger detonation waves.
    \item Injector design remains critical to achieve long term detonation stability and needs to be carefully chosen to meet the overarching goals for the RDE. Additionally, a proactive approach towards design for manufacturing and refurbishment capabilities is beneficial to maximize the testing time and the ability to recover from damage and erosion to internal engine components.
\end{enumerate}