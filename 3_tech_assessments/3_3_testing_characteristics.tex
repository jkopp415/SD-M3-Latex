\subsection{Testing Characteristics}

The purpose of this technology assessment is to provide background information on test characteristics and their impact on the project. This document will introduce the characteristics requiring testing, outline the fundamental principles of testing flight profiles, discuss relevant laws, standards, and physics behind simulating an air breathing engine. Additionally, it will examine the impact these factors have on the ready-to-integrate RDE project thus far. The contents of the following information is the guide that has been utilized in the requirements definition process, the design and information of our system architecture, and has facilitated further investigations into the testing of the RDE and related topics.

In the development of the SABR project thus far, we have remained cognizant of the ready-to-integrate standard for our engineering prototype of an RDE. We have further developed the proper testing characteristics that are essential to understand and simulate, as data acquisition and processing will allow us to develop an operational map of the complex and extreme operating conditions these engines face. RDEs function on the principle of continuous supersonic detonation waves, making them more efficient compared to traditional rocket engines. However, their unique combustion process introduces significant challenges, such as high thermal loads, pressure fluctuations, and detonation stability, all of which must be carefully assessed through testing.

Throughout much of the project cycle thus far, we have focused our attention on the flight profile and simulation capabilities of our system. We have based our design around a gas oxidizer fed system that will be diluted with nitrogen to achieve an air mixture ratio. To properly assess our experiment, we are comparing the inlet conditions with the dynamically changing atmosphere. Our rocket engine has been designed to withstand the constantly changing variables of the properties of the gas fed oxidizer. The flight profile presented plays a role with this selection of operation, as we want to be able to test operative capability at potentially different compositions of the atmosphere. Additionally, our exit geometry has been modified through the flight profile assessment. We sought to utilize a nozzle that is capable of operation at variable altitudes and therefore variable static pressures. As aircraft ascend in altitude, air density decreases and changes the pressure that the exhaust gases experience. This can greatly impact the efficiency of these engines; therefore, a consideration of nozzle types was determined to be necessary. Properly accounting for these changing factors is critical for optimizing the engine performance across various flight stages.