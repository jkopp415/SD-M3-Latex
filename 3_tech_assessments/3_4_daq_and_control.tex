\subsection{Data Acquisition and Control}

Data acquisition is the backbone of any experimental setup. The accuracy and reliability of the collected data determine the extent to which conclusions can be drawn from the experiments conducted. Strategic placement of thermocouples, pressure transducers, mass flow meters, flow controllers, load cells, and other instruments is required to evaluate and document critical test parameters and outcomes.  These sensors are read and translated using a DAQ device, each of which has a maximum sampling rate and a maximum number of channels that sensors can occupy. These parameters will determine how the DAQ is configured and what sensors can be utilized in this system.

Several factors play a critical role in this project and although all the topics presented in the technology study memo are important, one of the most vital of these is the sampling rate. This is because the rest of the parameters heavily rely on high-quality top-of-the-shelf sensors which is not a luxury we can afford. Sampling rate, on the other hand, can be easily changed and is set by the DAQ itself. The specific DAQ model we purchased (NI USB-6210) has the capability to do up to 250 kS/s. However, this is not 250 kS/s per channel, this is the total combined sampling rate across all channels. Due to the quick operation time and short duration of our tests, a high sampling frequency is recommended to observe the full scope of phenomena that are occurring within the system. In reality, sampling at a rate of 10 kS/s per channel would be more than sufficient for our sensors.

Another consideration when designing the data acquisition schematic is the number of DAQ channels. This is also a highly important topic as it determines how many total sensors we can utilize in our system. With the selected hardware (NI USB-6210) we have a total of 16 analog voltage inputs (±10 V), four (4) digital inputs, and four (4) digital outputs. The analog input channels are where the sensors will reside and thus are the most important here. The analog channels can be run in either a single-ended or differential configuration. Single-ended only utilizes one channel per sensor, using the common ground as the reference. While this allows the use of all 16 channels, it introduces a significant amount of noise in the resultant data. Differential, on the other hand, uses the common from the sensor as the reference voltage, thus producing a much cleaner signal but occupies two channels per sensor. After assessing the needs of each sub-system, we have decided to proceed with the differential configuration for all sensors. The total number of sensors we need at this stage is minimal and can be reduced to fit within the 8-channel maximum, five (5) pressure transducers for the feed lines and plenum to determine inlet mass flow rates, and three (3) load cells to measure the thrust output of the system.

With these DAQ restrictions in mind, the sensor selection may begin. However, given the budget constraints there will likely only be a few options. However, the following sensor parameters still greatly affect how the data acquisition is implemented and understood. Accuracy, resolution, range, sampling rate, and response time all play a critical role in sensor selection. Accuracy is defined as the maximum difference that will exist between the actual value and the indicated value at the output of the sensor \cite{ni-fundamentals:2023}. Sensors often provide users with an accuracy specification in which they are accurate within a range of values. However, in addition to this confidence interval, calibration, scaling, and signal shielding must be taken into account to produce accurate measurements.

The resolution of a sensor is the smallest detectable incremental change of input parameter that can be detected in the output signal \cite{ni-fundamentals:2023}. The resolution of a sensor is critical, if the incorrect sensor is selected, it may not pick up on small changes in the system and miss key findings. The range of a sensor determines its measuring limits. If a given sensor is exposed to conditions outside its measurement range, it will likely permanently damage the sensor. Thus, when selecting a sensor, analysis should be conducted in order to derive theoretical minimums and maximums that each sensor will see.

Similar to the resolution, sampling rate is highly important as it determines the level of granularity that can be seen in a system. A slow sampling rate may be adequate for monitoring some systems, but for systems with rapid variations like a RDE, a slow sampling rate would not capture the micro-second changes that are occurring in the system. Response time is defined by the time it takes a sensor to respond to a change in the system state. This reaction time, if too long, can cause the measurement system to completely miss quick phenomenon that are occurring.

Data acquisition is the most important thing when it comes to experimental testing. Without clear, concise, and reliable data the results of an experiment are left unsupported. To procure reliable data, the optimal selection of sensors must be completed by quantifying expected outputs and selecting sensors that adhere to the accuracy, resolution, range, sampling rate, and response time requirements of the system. These sensor selection parameters can be applied to the small-scale RDE system to evaluate critical measurements such as fuel flow rate, oxidizer flow rate, specific impulse, fuel injection parameters, and wave propagation speed.