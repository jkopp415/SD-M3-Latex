\subsection{RDE Combustor}

RDEs are a developing form of propulsion and power generation that utilize the radial propagation of detonation waves within an annulus to combust propellants. These combustors have been proven to work with a combination of solid, liquid, and gas fuels with liquid or gas oxidizers. They can run in either an air-breathing or rocket modes. Additionally, small-scale rocket RDE combustors have been demonstrated as feasible; however, much work remains to characterize their operation with different sizes, propellant combinations, and operating conditions due to the remaining work that needs to be conducted on detonation physics and how it changes in RDEs with high curvature.

Detonation is one of two forms of combustion (deflagration and detonation) that uses reacting, propagating shockwaves to combust reactants. The detonation thermodynamic cycle, the Zeldovich-von Neumann-Döring (ZND) cycle, shows a pressure gain during the heat addition process, which provides a greater work output. This principle is why RDEs are such an attractive technology. If a higher work output per propellant mass is achieved, it can decrease the cost and weight required for a system as well as its environmental impact.

The ZND detonation model is characterized by a compression wave followed by a reaction zone, then an expansion wave. When the compression wave interacts with reactants, there exists a spike in pressure, density, and temperature of the chemical species. This in turn induces high pressure combustion and volume expansion, which causes the compression wave to propagate. The wave then propagates at a theoretical velocity, called the Chapman-Jouguet (CJ) velocity. This velocity can be determined by applying the definition of the Mach number (Ma) to the upper CJ point, which is the intersection of the Rankine and Hugoniot lines.

In practice, the velocity experienced in RDEs is about 60\% of the CJ velocity.  The explanation as to why this occurs is still heavily debated, but it is theorized that it is attributed to parasitic deflagration, where reactants combust due to deflagration. Due to the detonation combustion not being present in all the propellant injected, there is not an ideal pressure rise. This causes a decrease in the pressure applied to the compression wave. Thus, it does not propagate at the ideal theoretical speed.

In aircraft engines, the performance parameters are typically the thrust generated, the specific impulse, the work output, and the thermodynamic efficiency. For a combustor, the means of optimizing these are through the achieving full combustion by optimizing parameters such as equivalence ratio, mass flow rate, and the dimensions of the combustor. However, there are some additional unique parameters that affect RDEs.

It has been shown that detonation cell size is correlated to the overall performance of RDEs. This is because if the detonation cell size is large compared to the annular gap, there will not be enough detonation cells to induce new cell formation and thus wave propagation. This cell size is highly dependent on the propellant mixture and the pressure of the mixture prior to wave interaction.

Various propellants have been tested in RDEs, especially small-scale RDEs. For the case of Project SABR, we will be using air as an oxidizer. Thus, the fuel chosen must optimize the key performance parameters of RDEs. The limiting parameter for our operation is the detonation cell size. The two fuels that have cell sizes small enough for a small-scale RDE are Hydrogen and Methane. From experimental data, it is shown that the cell size of Hydrogen is an order of magnitude smaller than Methane at our desired operating pressure, and thus will maximize our ability to achieve an ideal operation. Additionally, Hydrogen requires less energy input than Methane to induce detonation, and thus is a desirable choice.

Small-scale RDEs have been tested with a variety of designs and propellants. There are a series of theses from the Air Force Institute of Technology that have built on designs and reported challenges faced during operation. Additionally, they provide a template for small-scale RDE design for combustion, stabilization, injection, and material choice. Thus, their findings will play a heavy role in contextualizing our design approach.

Additionally, a three-inch RDE using air and Hydrogen was tested for operating condition limits in a collaboration between researchers in Beijing, China and Warsaw, Poland. It found stable detonation, quasi-stable detonation, unstable detonation, and fast-deflagration combustion for a variety of mass flow rates and pressure ratios. They also studied the detonation wave dynamics during their operation. Their findings for optimal operation will serve as guardrails in our desired flow properties.

Overall, detonation combustion is a very useful and efficient process of providing work to a system. The combination of physical, chemical, and thermodynamic models will lead to the definition of key performance parameters. This analysis can be complicated, but by using and customizing proven methods, it can be done relatively easily. Applying the known detonation properties of fuels to key performance parameter definitions and related equations, the resultant values should be compared to inform the decision on which fuel to use. Additionally, the phase at which the propellants are injected into the combustor may have an impact on performance, and thus should be taken into consideration during the analysis and/or decision-making process. Using proven methods of analysis, reviewing past successes and failures, and the effect of performance parameters on other design choices will assist in improving the quality of the system and the engineering process.