\subsection{Test Stand Configuration}

The purpose of this technology assessment is to provide background information on the design requirements for a test stand tailored to small-scale air-breathing rotating detonation engines and their impact on the project as a whole. This document will outline the essential principles of test stand design and assembly, describe relevant technologies for thrust measurement, and explore their implications for the successful testing of RDEs. The following information has guided the process of further defining our project requirements, informed system architecture development, and enabled further investigations into the testing and evaluation of test stand technology.

Over the course of the project thus far, the focus has been on designing a test stand capable of meeting the structural, measurement, and operational demands that are unique to RDEs. Unlike conventional rocket engines, RDEs generate thrust though supersonic detonation waves, which introduces many more unique challenges to the process of designing and developing an adequate test stand for this use. These factors require a specialized approach to test stand design, enabling reliable structural stability and high accuracy thrust data collection over the course of multiple tests. As we continue to work towards a functional hot fire test of a small-scale RDE, we have focused our attention down to two primary areas when it comes to test stand development: the stand must provide stability to minimize the effects of external forces on the engine’s performance and must be able to measure the thrust produced by the engine to a high level of accuracy. By addressing these core areas, the test stand will be able to serve as a reliable platform for characterizing the performance and operational behavior of the RDE.

One of the most obvious and important features for the test stand to exhibit is the ability to combine robustness with adaptability, being able to support the engine under extreme operating conditions. Throughout much continued research, two frequently used techniques for structurally supporting the engine emerged, serving as great methods for this task while remaining within the manufacturing and budgetary limits constraining the project. One such method is to mount the engine on a set of linear rails, which would in turn constrain the engine’s motion to a single linear axis, better directing thrust forces into load cells for accurate measurement. This idea has remained very popular in test stand design due to its ability to nearly eliminate all external lateral forces, which would otherwise distort thrust measurements. Another popular design is to mount the engine to a “floating” plate that is only connected to the rest of the test stand’s static structure via the load cells. This solution achieves the same effect of eliminating external lateral forces almost as well as the rail-based design but comes with the added benefit of being much cheaper and easier to fabricate.

Another extremely important feature that the test stand must include is a system to calibrate load cells, something that must be done prior to testing to ensure that the thrust data being collected is accurate. There were two popular methods for this that emerged as well, both with similar benefits and limitations, making the choice between the two options rely on external project limitations. One method is to utilize a hydraulic ram and associated static structure which can attach to the load transfer structure, or even the RDE itself, and apply a known force to test and calibrate the load cells. A similar result can be obtained with the much simpler, but less efficient, mass and pulley calibration system in which weights of known mass can be attached to a cord running over a pulley and attaching to the back of the load transfer structure in order to calibrate the load cells. With these multiple options, it is absolutely crucial to ensure that the load cells are able to accurately measure the exact thrust being produced by the RDE.

While the specific configurations of the test stand may not entirely control the development of the RDE system on its own, the choices made to achieve the two important test stand functions do have a large impact on the rest of the project, mainly though budgetary, manufacturing, and timeline restrictions. For each configuration option for the test stand, there is a balance between data accuracy and cost/complexity, and in many cases due to the timeline and budget of the project, the slightly cheaper and less complex option may be chosen, as long as it can be ensured that the main project requirements are still being met. In addition, the different options have a large impact on the adaptability of the test stand, as some configurations may be able to better accommodate future modifications to the RDE at the cost of more expensive and complicated designs.

In conclusion, while the test stand is not critical to the function of the RDE itself, it is extremely crucial to the successful development, evaluation, and implementation of RDE technology. By continuing to narrow and address the most important functions of the test stand, the ability to thoroughly test the engine and its unique operational characteristics will provide plenty of data to inform the design, optimization, and validation of RDE systems. Drawing from these existing designs will allow the test stand to support the rigorous demands of RDE testing and contribute to the overall success of this project.