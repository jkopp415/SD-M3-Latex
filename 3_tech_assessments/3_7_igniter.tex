\subsection{Igniter}

Deflagrations and detonations are both mechanisms of combustion by which a mixture of fuel and an oxidizer release heat into a system. The fundamental difference being that deflagrations are associated with flame fronts propagating at subsonic velocities (in the context of rocket engines, the flame front is equal to the injection velocity which is usually on the order of 10-100 meters per second) while detonations propagate at supersonic velocities, often on the order of several to tens of kilometers per second, depending on the mixture chemistry and density. Because detonations propagate at supersonic speeds, a shock wave forms in front of the combustion zone which compresses and raises the temperature of the fuel-oxidizer mixture in front which then detonates and continues the reaction. It should be noted that the wave speed and pressure rise associated with detonations is proportional to the density of the reactants. For reference, gas/gas reactants can yield pressure rises in the several or tens of thousands of psi, while liquid/liquid reactants can easily achieve detonation pressure on the order of millions of psi and significantly higher detonation velocities \cite{heister:2022}.

The transition from deflagration to detonation and the following propagation of detonation waves depends on several factors, but almost always driven by how properly mixed the reactants are in a control volume. If we imagine a long tube filled with an air-fuel mixture that is then ignited by some external means, the resulting flame will initially travel at subsonic speeds through the tube. As the deflagration propagates, it begins to accelerate due to minor gains in velocity from shock reflection, shock focusing, or instabilities/interaction between the flame front and mach waves \cite{breitung:2000} (an infinitely weak shock wave, if enough mach waves stack upon each other due to compressible effects, it creates a shock wave). If the deflagration accelerates long enough it will eventually evolve into a full detonation wave. However, this process requires a tube of considerable L/D ratio so the characteristic length must be long enough for the deflagration to accelerate.

Pre-detonation igniters are devices that take advantage of DDT phenomena to ignite RDEs. They work by filling a long tube with reactants (usually the same mixture being fed into the main chamber, just tapped off the main lines) and ignited via spark ignition. The mixture will then undergo DDT, generating a detonation wave by the time the flame front leaves the tube and enters the combustor. Because pre-detonation igniters generate a shock, only a single firing is required to startup an RDE. However, they do require a separate feed and valve system which poses some challenges for integration and packaging constraints.

Deflagration to detonation phenomenon gives us several avenues towards the development of a reliable ignition method for our proposed engine. A pre-detonation ignition system works by filling a tube connected to the combustor with reactants, which is then ignited via spark igniter. The reactants eventually undergo a transition from deflagration to detonation given enough tube length where the resulting detonation wave is fed into the combustor. This provides the engine with a strong initial detonation wave and has been experimentally observed to be a more reliable method of engine startup. A direct-spark ignition system is more compact than a pre-detonation system as it does away with the relatively long length of tube required to initiate a deflagration to detonation transition. A direct-spark ignition system involves directly threading the spark plug into the combustor, using the detonation channels as the characteristic length needed to promote a detonation transition. While it is a much simpler method, a recessed spark plug has been experimentally shown to be unreliable, often failing to transition the deflagration flame front to a detonation. On the other hand, a spark plug whose electrode is extruding into the chamber suffers from erosion due to constant exposure to the detonation waves \cite{dechert:2020} and requires replacement after only several firings. These characteristic strengths and weaknesses must be considered in the design process of our engine, as a poor ignition method can hamper the progress during our test campaign and restrict the cadence of firings.